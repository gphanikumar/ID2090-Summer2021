\documentclass{article}
\documentclass[a4paper, 12pt]{article}
\usepackage[left=2cm, right=2cm, top=2cm, bottom=2cm]{geometry}
\setlength{\parindent}{0cm}
\usepackage{graphicx}
\usepackage{amsmath}
\usepackage[colorlinks=true, allcolors=blue]{hyperref}

\begin{document}

\author{Sai Karthik Brahma}

\title{ID2090 Assignment 4}

\maketitle

\begin{abstract}
This paper consists my favourite equations in Physics.
\end{abstract}
\tableofcontents


\section{Introduction to the Maxwell's Equations}

Maxwell's Equations are a set of 4 complicated equations that describe the world of electromagnetics. These equations describe how electric and magnetic fields propagate, interact, and how they are influenced by objects.

James Clerk Maxwell genius who took a set of known experimental laws (Faraday's Law, Ampere's Law) and unified them into a set of Equations known as Maxwell's Equations. 

Maxwell was one of the first to determine the speed of propagation of electromagnetic (EM) waves was the same as the speed of light - and hence to conclude that EM waves and visible light were really the same thing.

An important consequence of \href{https://en.wikipedia.org/wiki/Maxwell's_equations} {Maxwell's equations} is that they demonstrate how fluctuating electric and magnetic fields propagate at a constant speed (c) in a vacuum. Known as electromagnetic radiation, these waves may occur at various wavelengths to produce a spectrum of light from radio waves to gamma rays.


\subsection{Maxwells Equation's}

Maxwell's Equations are laws - just like the law of gravity. These equations are rules the universe uses to govern the behavior of electric and magnetic fields


\begin{equation}
\Vec{\nabla}.\Vec{E} = \dfrac{\rho}{\epsilon_0}
\end{equation}

\begin{equation}
\Vec{\nabla}.\Vec{B} = 0
\end{equation}

\begin{equation}
\vec{\nabla}.\times\vec{E} = -\dfrac{\partial B}{\partial t}
\end{equation}

\begin{equation}
\vec{\nabla}.\times\vec{B} = \mu_0\vec{J} + \dfrac{1}{c^2}\dfrac{\partial E}{\partial t}
\end{equation}


\end{document}
