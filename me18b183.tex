\documentclass[a4paper, 11pt]{article}
\usepackage{comment} % enables the use of multi-line comments (\ifx \fi) 
\usepackage{lipsum} %This package just generates Lorem Ipsum filler text. 
\usepackage{fullpage} % changes the margin
\usepackage[a4paper, total={7in, 10in}]{geometry}
\usepackage[fleqn]{amsmath}
\usepackage{amssymb,amsthm}  % assumes amsmath package installed
\newtheorem{theorem}{Theorem}
\newtheorem{corollary}{Corollary}
\usepackage{graphicx}
\usepackage{tikz}
\usetikzlibrary{arrows}
\usepackage{verbatim}
\usepackage[numbered]{mcode}
\usepackage{float}
\usepackage{tikz}
    \usetikzlibrary{shapes,arrows}
    \usetikzlibrary{arrows,calc,positioning}

    \tikzset{
        block/.style = {draw, rectangle,
            minimum height=1cm,
            minimum width=1.5cm},
        input/.style = {coordinate,node distance=1cm},
        output/.style = {coordinate,node distance=4cm},
        arrow/.style={draw, -latex,node distance=2cm},
        pinstyle/.style = {pin edge={latex-, black,node distance=2cm}},
        sum/.style = {draw, circle, node distance=1cm},
    }
\usepackage{xcolor}
\usepackage{mdframed}
\usepackage[shortlabels]{enumitem}
\usepackage{indentfirst}
\usepackage{hyperref}
    
\renewcommand{\thesubsection}{\thesection.\alph{subsection}}

\newenvironment{problem}[2][Problem]
    { \begin{mdframed}[backgroundcolor=gray!20] \textbf{#1 #2} \\}
    {  \end{mdframed}}

% Define solution environment
\newenvironment{solution}
    {\textbf{\textit{Solution:}}}
    {}

\renewcommand{\qed}{\quad\qedsymbol}
%%%%%%%%%%%%%%%%%%%%%%%%%%%%%%%%%%%%%%%%%%%%%%%%%%%%%%%%%%%%%%%%%%%%%%%%%%%%%%%%%%%%%%%%%%%%%%%%%%%%%%%%%%%%%%%%%%%%%%%%%%%%%%%%%%%%%%%%
\begin{document}
%Header-Make sure you update this information!!!!
\noindent
%%%%%%%%%%%%%%%%%%%%%%%%%%%%%%%%%%%%%%%%%%%%%%%%%%%%%%%%%%%%%%%%%%%%%%%%%%%%%%%%%%%%%%%%%%%%%%%%%%%%%%%%%%%%%%%%%%%%%%%%%%%%%%%%%%%%%%%%
\large\textbf{SHINDE SHUBHAM SUNIL} \hfill \textbf{ASSIGNMENT \#2}   \\
\textbf{smail:} me18b183@smail.iitm.ac.in \hfill \textbf{Roll No.:} ME18B183 \\
\normalsize\textbf {Course:} ID2090 - Introduction to Scientific Computing \hfill \textbf{Semester:} SUMMER 2021 \\
\textbf{Instructor:} Prof. Gandham Phanikumar \hfill \textbf{Due Date:} $17^{th}$ July, 2021 \\
\noindent\rule{7in}{2.5pt}
%%%%%%%%%%%%%%%%%%%%%%%%%%%%%%%%%%%%%%%%%%%%%%%%%%%%%%%%%%%%%%%%%%%%%%%%%%%%%%%%%%%%%%%%%%%%%%%%%%%%%%%%%%%%%%%%%%%%%%%%%%%%%%%%%%%%%%%%
% Problem 1
%%%%%%%%%%%%%%%%%%%%%%%%%%%%%%%%%%%%%%%%%%%%%%%%%%%%%%%%%%%%%%%%%%%%%%%%%%%%%%%%%%%%%%%%%%%%%%%%%%%%%%%%%%%%%%%%%%%%%%%%%%%%%%%%%%%%%%%%
\begin{problem}{1}
Use the \$RANDOM shell variable that provides a random number between 0 and 32767 to generate a csv file that will contain marks for a set of 50 students. Scale or use mod operation to get the range correct. The roll number can be a serial number from 1 to 50. The marks in percentage shall be in three columns, namely, Q1 (max 25), Q2 (max 25) and EndSem (max 50). Name this csv file as “marks.csv”. Using a separate script, run through these 50 rows to normalize the marks (topper is scaled to 100) and to assign grades in the following rubric: S for marks $>$90; A for $>$80; B for $>$70; C for $>$60; D for $>$50; E for $>$40 and U for 40 or below. Print the output showing the original mark break up, normalized mark and the grade against each student.\\
\textit{\textbf{Output Required:} The two codes–one to create the csv and one to analyze it to assign grades; the csv file itself; the output file showing the mark break up, normalized marks and the grades.}
\end{problem}
\begin{solution}


\end{solution} 
%\noindent\rule{7in}{2.5pt}


{\huge \centerline{\ast \ast \textbf{End of Assignment} \ast \ast}}
\\
\end{document}
 
