\subsection{De Broglie Equation}

{

If I were to choose an equation in physics, I’d definitely go for the famous De Broglie Equation.

De Broglie was a French physicist and aristocrat who made ground-breaking contributions to quantum theory. In his 1924 PhD thesis, he postulated the wave nature of electrons and suggested that all matter has wave properties. This concept is known as the de Broglie hypothesis, an example of wave–particle duality, and forms a central part of the theory of quantum mechanics.

\begin{equation}

\p = frac{\h}{\lambda}

\label{De Broglie Equation}

\end{equation}

}

\subsection{Terms and their meanings}

{

\begin{itemize}

\item $\p :$ The momentum of the particle.

\item $\h :$ The Planck’s Constant.

\item $\lambda :$ The De Broglie wavelength.

\end{itemize}

}

\subsection{The De Broglie Equation : A short Insight}

{

The de Broglie equation is an equation used to describe the wave properties of matter, specifically, the wave nature of the electron. De Broglie suggested that particles can exhibit properties of waves which was later verified by cathode ray diffraction experiment and the Davisson-Germer experiment.

}
