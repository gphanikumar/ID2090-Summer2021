\begin{center}
    \textbf{Acceleration of a body in a Rotating Frame}
\end{center}
$$ \vec{a} = \vec{a}_o + \vec{\alpha} \times \vec{r}_{rel} + \vec{\omega} \times \vec{\omega} \times \vec{r}_{rel} + 2 \vec{\omega} \times \vec{v}_{rel} + \vec{a}_{rel} $$

where the above used symbols stand for
\begin{itemize}
    \item $ \vec{r}_{rel} $: position vector of object with relative to the rotating frame
    \item $ \vec{v}_{rel} $: velocity of object relative to the rotating frame
    \item $ \vec{a}_{rel} $: acceleration of object relative to the rotating frame
    \item $ \vec{a}_o $: absolute acceleration of the rotating frame
    \item $ \vec{\omega} $: angular velocity of the rotating frame
    \item $ \vec{\alpha} $: angular acceleration of the rotating frame
\end{itemize}

This equation may not be among the flashy equations of Physics, like \textit{Navier-Stokes equation}, \textit{Schrodinger's equation}, etc., but is definitely one of my favourite equations out there. My personal affinity to this equation particularly comes from the presence the \textbf{Coriolis acceleration} term ($ 2 \vec{\omega} \times \vec{v}_{rel} $). When I first learnt about this concept of Coriolis acceleration in my 1st semester at IITM, it intrigued me to the point of watching numerous YouTube videos to understand what this \textit{Coriolis acceleration} actually meant. I went on the explore its significance in day-today life and found out that the direction of rotation of a hurricane/cyclone is determined by the Coriolis acceleration provided by the rotating Earth! I can very confidently say that this concept and equation kindled my inner Mechanical Engineer and has kept me attached to my subject ever since.
