\subsection{Schrödinger Equation}

{
    If given a choice to pick on equation in Physics, I would pick the Schrödinger Equation.
    The equation is named after Erwin Schrödinger, who postulated the equation in 1925. Schrödinger equation is the fundamental equation of quantum mechanics, although not the only way to study quantum meachanical systems, it still holds it's elite place and respect  among the physics community. Conceptually, the Schrödinger equation is the quantum counterpart of Newton's second law in classical mechanics, thus no surprise that they are often compared.

    \begin{equation}
        \hat{H} \Psi = i\hbar\frac{\partial\Psi}{\partial t}
        \label{Schrödinger Equation}
    \end{equation}
}

\subsection{Terms and their meanings}
{
    	\begin{itemize}
    
        \item $\Psi(x,t) :$ The Wave function, a function that assigns a complex number to each point ${\displaystyle x}$ at each time ${\displaystyle t}$.
        
        \item $\hbar :$ The reduced Planck's constant. 
        
        \item $\hat{H} :$ The Hamiltonian operator.
        
        \end{itemize}
}

\subsection{The Schrödinger Equation : An overview}
{
    The Schrödinger Equation is the fundamental equation of study of submicroscopic phenomina, known as Quantum Mechanics. Essentially a wave equation, the Schrödinger Equation describes the probability waves, that are responsible for motion of submicroscopic partices.
    Erwin Schrödinger proved the correctness of equation by it's application on Hydrogen atom, predicting many of it's properties with great accuracy. The applicaion of Schrödinger Equation is remarkable in atomic, nuclear and solid-state physics.    
}
