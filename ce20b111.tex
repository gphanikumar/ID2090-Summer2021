\documentclass[]{article}
\begin{document}

\section{Favorite equation from physics}
\begin{center}
\Large{\textbf{\underline{Newton's Law of Gravitation}}}
\end{center}
\begin{justify}
Acoording to Newton's law of universal gravitation, every particle in the universe attracts every other particle with a force which is directly proportional to the product of their masses and inversely proportional to the square of the distance between their centers.
\end{justify}
\begin{equation}
  F = G \times \frac{m_1 m_2}{r^2}  
  \end{equation}
\begin{itemize}
  \item $F$ is the force between the masses
  \item $G$ is the Universal Gravitational constant and its value is $6.674 \times 10 ^{-11}  kg^{-1}m^{3}s^{-2}$
  \item $m_1$ and $m_2$ are the masses of the particles
  \item $r$ is the distance between the centers of the masses.
\end{itemize}


\end{document}

