\subsection{Time Dependant Schrödinger Equation}


$$ i \hbar \frac{\partial}{\partial t}\Psi(\mathbf{r},t) = \hat H \Psi(\mathbf{r},t) $$


The form of the Schrödinger equation depends on the physical situation. The most general form is the time-dependent Schrödinger equation, which gives a description of a system evolving with time

The term "Schrödinger equation" can refer to both the general equation, or the specific nonrelativistic version. The general equation is indeed quite general, used throughout quantum mechanics, for everything from the Dirac equation to quantum field theory, by plugging in diverse expressions for the Hamiltonian. 

The specific nonrelativistic version is an approximation that yields accurate results in many situations, but only to a certain extent (see relativistic quantum mechanics and relativistic quantum field theory).

\begin{itemize}
    \item $ \hbar $ : Planck's constant divided by $ 2\pi $
    \item $ \hat H $ : Observable Hamiltonian operator
    \item $ \Psi(\mathbf{r},t) $ : State vector of the quantum system
    \item $ t $ : Time
\end{itemize}
