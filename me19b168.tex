\subsection{Heisenberg's Uncertainty Principle}
{
    My favorite equation in Physics is Heisenberg's Uncertainty Principle. This is because, even though it is a very simple and concise inequality, the implications it has is very vast. The equation is written as:

    \begin{equation}
        \Delta x \Delta p \geq {\hbar \over 2} 
        \label{Uncertainty Principle}
    \end{equation}
}
Where:
\begin{itemize}
    \item $\Delta x$ = Uncertainty in position
    \item $\Delta p$ = Uncertainty in momentum 
    \item $\hbar$ is the Reduced Planck's Constant and is equal to $h \over 2\pi$ where $h$ is the Planck's Constant
\end{itemize}

\subsection{Meaning of Uncertainty Principle}
{
    Heisenberg's Uncertainty Principle states that both the momentum and the position of a particle cannot be obtained with 100\% accuracy, i.e. the product of the error in the measurement of position and measurement of momentum, $\Delta x \Delta p$, is always greater than or equal to the value $\hbar \over 2$ (or $h \over 4\pi$).
}
