\section{Maxwell's Laws}
Maxwell's laws are a collection of four equations that were brought together and used to formulate a unified description of electricity and magnetism by Scottish physicist \textbf{James Clerk Maxwell} in 1862. The equations provide a mathematical model for electric, optical, and radio technologies, such as power generation, electric motors, wireless communication, lenses, radar etc. They describe how electric and magnetic fields are generated by charges, currents, and changes of the fields.  


\subsection{Gauss's law}
The relationship between a static electric field and electric charges is described by Gauss's law (Equation~\ref{eqn:GL}), which states that a static electric field points away from positive charges and towards negative charges, and that the net outflow of the electric field({E}) through a closed surface is proportional to the enclosed charge, including bound charge due to material polarization. The coefficient of the proportion is the permeability of free space.

\begin{equation}
\label{eqn:GL}
{\nabla} \cdot {E} = \frac{\rho}{\varepsilon_0}
\end{equation}
where:
\begin{center}
${\rho}$ = total electric charge density\\
${\varepsilon_0}$ = Permittivity of free space\\
\end{center}

\subsection{Gauss's law for magnetism}
Gauss's law for magnetism(Equation~\ref{eqn:GML}) states that electric charges have no magnetic analogues, called magnetic monopoles. Instead, the magnetic field({B}) of a material is attributed to a dipole, and the net outflow of the magnetic field through a closed surface is zero. Magnetic dipoles can be thought of as current loops or inseparable pairs of equal and opposite magnetic charges. In a Gaussian surface, the total magnetic flux is zero, and the magnetic field is a solenoidal vector field. 

\begin{equation}
\label{eqn:GML}
{\nabla} \cdot {B} = 0
\end{equation}


\subsection{Faraday's law}
Faraday's law of induction is described in the Maxwell–Faraday Equation(Equation~\ref{eqn:FL}), which explains how a time-varying magnetic field({B}) generates an electric field({E}). It states that the work per unit charge required to move a charge around a closed loop equals the rate of change of the magnetic flux through the enclosed surface.
\begin{equation}
\label{eqn:FL}
{\nabla} \times {E} = -\frac{\partial {B}}{\partial t}
\end{equation}

\subsection{Ampere's circuital law (with Maxwell's addition) }
Magnetic fields can be formed in two ways, as per Ampere's law with Maxwell's addition(Equation~\ref{eqn:AML}): by electric current and by changing electric fields({E}).
The magnetic field({B}) induced around any closed loop is proportional to the electric current({J}) + displacement current(addition by Maxwell) through the enclosed surface (proportional to the rate of change of electric flux). 

\begin{equation}
\label{eqn:AML}
{\nabla} \times {B} = \mu_0\left({J}+{\varepsilon_0}\frac{\partial {E}}{\partial t}\right)
\end{equation}
where:
\begin{center}
${\varepsilon_0}$ = Permittivity of free \\
$\mu_0$ =  permeability free space\\
\end{center}