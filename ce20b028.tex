\subsection{Einstein's Energy-Mass Equivalence  }
The formula defines the energy E of a particle in its rest frame as the product of mass  with the speed of light squared . Because the speed of light is a large number in everyday units (approximately 3×108 meters per second), the formula implies that a small amount of rest mass corresponds to an enormous amount of energy, which is independent of the composition of the matter. Rest mass, also called invariant mass, is the mass that is measured when the system is at rest. It is a fundamental physical property that is independent of momentum, even at extreme speeds approaching the speed of light (i.e., its value is the same in all inertial frames of reference). Massless particles such as photons have zero invariant mass, but massless free particles have both momentum and energy. The equivalence principle implies that when energy is lost in chemical reactions, nuclear reactions, and other energy transformations, the system will also lose a corresponding amount of mass. The energy, and mass, can be released to the environment as radiant energy, such as light, or as thermal energy. The principle is fundamental to many fields of physics, including nuclear and particle physics.
\begin{equation}
    E=\frac{mc^2}{\sqrt{1-v^2/c^2}}
    \label{Energy-Mass Equivalence Relation}
\end{equation}
where
\begin{itemize}
    \item[$E$] is the Energy in matter
    \item[$m$] is the mass of matter
    \item[$v$] is the velocity of matter
    \item[$c$] is the velocity of speed in vaccum
\end{itemize}

