
\subsection{Biot–Savart law}
In physics, specifically electromagnetism, the Biot–Savart law is an important and is one of my \textbf{favourite equation} describing the magnetic field generated by a constant electric current. It relates the magnetic field to the magnitude, direction, length, and proximity of the electric current. The Biot–Savart law is fundamental to magnetostatics and is represented by Equation~\ref{eqn:BSL}:
\begin{equation}
\label{eqn:BSL}
d\vec{B}_P=\frac{\mu_0\mu_rI}{4\pi r^2}d\vec{l}\times\vec{r}
\end{equation}
where:\\
$\vec{B}_P$  is resultant magnetic field at position r\\
$\mu_0$  is permeability of free space\\
$\mu_r$  is the relative permeability of the medium\\
$d\vec{l}$  is a vector along the path whose magnitude is the length of the      differential element of the wire in the direction of conventional current\\
I  is the current flowing through the conductor\\
$\vec{r}$  is the vector from the element of wire
