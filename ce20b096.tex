\documentclass[a4paper, 12pt]{article}
\usepackage[left=2cm, right=2cm, top=2cm, bottom=2cm]{geometry}
\usepackage{graphicx}
\title{My Favourite Physics Equation}
\author{Sambhu J CE20B096}
\begin{document}
\maketitle
My favourite physics equation is the Maxwell-Faraday equation which is one among the four well-known Maxwell's equations. It states that a time-varying magnetic field always accompanies a spatially varying (also possibly time-varying), non-conservative electric field, and vice versa.

The equation in differential form is :

$$ \vec{\nabla} \times \vec{E} = -{\partial \vec{B} \over \partial dt} $$

where $ \vec{\nabla} \times $ is the curl operator and $ \vec{E}(r, t) $ is the electric field and $ \vec{B}(r, t) $ is the magnetic field. These fields can generally be functions of position r and time t. 

The equation in integral form is :

$$ \oint_{\partial S} {\vec{E} \cdot d \vec{l}} = \int_{S} {{\partial \vec{B} \over \partial t} \cdot d \vec{A}} $$

where S is a surface bounded by the closed contour $ \partial S , d \vec{l} $ is an infinitesimal vector element of the contour $ \partial S $, and $ d \vec{A} $ is an infinitesimal vector element of surface S. Its direction is orthogonal to that surface patch, the magnitude is the area of an infinitesimal patch of surface.

\end{document}