

\begin{center}
Gauss's law of Magnetism
\end{center}
\begin{flushleft}
 Gauss's law for magnetism is one of the four Maxwell's equations that underlie classical electrodynamics. It states that the magnetic field B has divergence equal to zero, in other words, that it is a solenoidal vector field. It is equivalent to the statement that magnetic monopoles do not exist.
\end{flushleft}

\begin{equation}\label{Integ}
    \oint B\cdot d\overrightarrow{\! S}= 0
    \end{equation}
\begin{center}
Integral Form
\end{center}

\begin{equation}\label{Diff}
    \nabla \cdot{B} = 0
\end{equation}
\begin{center}
Differential Form
\end{center}





