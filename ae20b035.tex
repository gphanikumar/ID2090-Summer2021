

\begin{equation}
 \vec{\nabla}\times \vec{E} =  -\frac{d\vec{B}}{dt}
 \label{eq:1}
\end{equation}


    The Maxwell-Faraday equation (\ref{eq:1})   : 
    \begin{center}You can create a changing electric field $\vec{E}$(left side of the equation) from a changing magnetic field $\vec{B}$ (on the right) and vice versa with respect to time $t$.
\end{center}

\begin{equation}
 i\hbar\frac{\partial}{\partial t}\Psi(\textbf{r},t)=[{\frac{-\hbar^2}{2\mu}\nabla^2 +V(\textbf{r},t)}]\Psi(\textbf{r},t)\label{eq:2}
\end{equation}

The Schrödinger wavefunction (\ref{eq:2})   :  
\begin{center} It describes how the change of a particle’s wavefunction (represented by $\Psi$) can be calculated from its kinetic energy (movement) and its potential energy $V$ (the interactions on it). $\hbar$ is reduced Planck's constant. $\mu$ is reduced mass. $\textbf{r}$ and $t$ represents position vector and time respectively.
\end{center}



\begin{equation}
 \frac{\partial^2u}{\partial t^2}=v^2\nabla^2u 
 \label{eq:3}
\end{equation}


     The Wave Equation (\ref{eq:3})   : 
    \begin{center}The wave equation is a 2nd order partial differentiation equation that describes the propagation of waves.It relates the change of propagation of the wave in time ($t$) to the change of propagation in space and a factor of the wave speed ($v$) squared. Here $u$ the wave function such as displacement or pressure difference ,etc.
\end{center}
