\documentclass[a4paper, 12pt]{article}
\usepackage[left=2cm, right=2cm, top=2cm, bottom=2cm]{geometry}
\setlength{\parindent}{0cm}
\usepackage{graphicx}
\begin{document}

\title{Assignment 4}
\author{Nitya Nanvani ME19B138}

\maketitle

\section{Gauss's flux theorem}

\begin{equation}
    \oint_S {E_n dA = \frac{1}{{\varepsilon _0 }}} Q_{in}
\end{equation}

\begin{itemize}
	\item $ E_n $  refers to the component of the electric field, normal to the area.
	\item $ dA $  refers to the differential area which is integrated over a closed surface.
	\item $ \varepsilon _0 $ is the vacuum permittivity.
	\item $ Q_{in} $ refers to the total net charged enclosed inside the closed surface.
\end{itemize}
Gauss Law states that the total electric flux out of a closed surface is equal to the charge enclosed divided by the permittivity. The electric flux in an area is defined as the electric field multiplied by the area of the surface projected in a plane and perpendicular to the field.
\end{document}
