\begin{Large}

\begin{center}
\underline{Heisenberg Uncertainty Principle (1927)}
\end{center}
\begin{equation}
\Delta x\Delta\rho\quad\ge\quad{\hbar \over 2}
\end{equation}

\end{Large}

$\Delta x$ = Uncertainty in Position\\
$\Delta\rho$ = Uncertainty in Momentum\\
$\hbar={{h}/{2\pi}}$ ,\\
where $h$ = Planck's Constant and $\pi$ = pi (approx. 3.14159)
\\\\
Uncertainty principle, also called Heisenberg uncertainty principle or indeterminacy principle, statement, articulated (1927) by the German physicist Werner Heisenberg, that the position and the velocity of an object cannot both be measured exactly, at the same time, even in theory. The very concepts of exact position and exact velocity together, in fact, have no meaning in nature.







