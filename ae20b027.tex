\section{EQUATION AND DESCRIPTION}

\subsection{Forced Vibrations General Equation}

\begin{equation}

    m\frac{d^2}{dt^2}(x)+\beta\frac{d}{dt}(x)+kx=F(t)

\end{equation}

\subsection{Description}

\begin{center}

m = mass of object attached to the spring

\newlines

$\beta$ = damping constant

\newline

x = displacement

\newline

k = spring constant

\newline

F(t) = External force varying with time

\end{center}

Forced vibrations is a very interesting area in physics. It is a very important topic in the domain of mechanical, civil, metallurgy, aerospace and many other domain of engineering. The first approach for inspecting a forced vibration is through equation (1). This equation describes the motion of the mass attached to the spring in space when a time-varying force acts on it. This is a general equation. By solving this second order non-homogeneous differential equation we can find many the properties related to the mass m in this system such as amplitude and phase.













