\section{AE19B018}
\subsection{Schr\"{o}dinger's Equation}
{
My favorite equation in Physics is Schr\"{o}dinger's Equation. This equation fascinates me a lot because of the very important role it plays. 

\textbf{It has two different variations of the equation, one being time-dependent and the other being time independent.}
    
\subsubsection{Time Dependent Schr\"{o}dinger's Equation :}
    \vspace{5mm}
    \begin{equation}
       \hat{H} | \psi(t) = \hat{E} | \psi(t) 
    \end{equation}

Where,
\begin{itemize}
    \item $\psi$ (the Greek letter psi) is the state vector of the quantum system
    \item $E$ is the energy of the system.
    \item $\hat{H}$ is an observable, the \textbf{Hamiltonian operator.}
    \item $t$ is the time
\end{itemize}

\subsubsection{Time Independent Schr\"{o}dinger's Equation :}
\vspace{5mm}
\begin{equation}
       i\hbar\frac{d}{dt}|\psi(t) = \hat{E} | \psi(t)
    \end{equation}

Where,
\begin{itemize}
    \item $\psi$ (the Greek letter psi) is the state vector of the quantum system
    \item $E$ is the energy of the system.
    \item $t$ is the time
\end{itemize}

The time-dependent Schr\"{o}dinger equation described above predicts that wave functions can form standing waves, called stationary states. These states are particularly important as their individual study later simplifies the task of solving the time-dependent Schr\"{o}dinger equation for any state.

$\psi$ or the wave function, is the most important part of the Schr\"{o}dinger equation: the answer. The solution to the Schr\"{o}dinger equation is intended to describe the electron, and there can be an infinite number of solutions. However, the focus here is on $\psi$, which is :
\begin{equation}
    \psi = \sqrt{\pi e^{-r}}
\end{equation} 
where, $r$ is related to the distance from the center of the atom.
\vspace{5mm}
For example, if the equation is used for a hydrogen atom, many things about it can be predicted: the radius of the atom and the wavelengths of light that hydrogen emits. What psi shows is actually the potential place where the electron could be.
}