\documentclass[a4paper,12pt]{article}
\usepackage[left=2cm, right=2cm, top=2cm, bottom=2cm]{geometry}
\begin{document}
\title{Assignment4}
\date{July 24, 2021}
\author{Rohan Nakade AE20B036}
\maketitle
\section{The Bernaulli's equation}
\begin{equation}
P + \frac{1}{2}\rho v^2 + \rho gh = constant
\label{eqn:1}
\end{equation}

\section{Inroduction}
Being an aerospace department student I will say the Bernaulli's equation is the most useful and interesting equation without which it would be hard to advance in aerodynamics. 

In equation ~\ref{eqn:1}, the terms used are $ \rho $, $ v $, $ g $, $ h $ and $ P $.
\begin{itemize}
\item $\rho$ - density of fluid
\item $v$ - velocity of fluid
\item $g$ - accceleration due to gravity
\item $h$ - height of fluid flow in consideration from datum
\item $P$ - pressure of fluid at that point
\item $\frac{1}{2}\rho v^2$ - kinetic energy of fluid per volume
\item $\rho gh$ - potential energy of fluid per volume
\end{itemize}


\section{Explanation}
The equation is based on the principle of energy conservation, applied to flowing fluid.The Bernaulli's principle states that, in a steady flow, the sum of all forms of energy in a fluid along a streamline is the same at all points on that streamline.In other words the sum of kinetic energy, potential energy and internal energy remains constant.This equation has a very wide application in the field of fluid dynamics.
\end{document}
