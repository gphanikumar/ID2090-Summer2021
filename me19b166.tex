\subsection{The Dirac Equation}
{
    My favourite equation in Physics, if I could pick just one, would have to be the Dirac Equation. The Dirac Equation wholeheartedly deserves the capitalization used here, as it was one of the first successful steps towards unification. In 1928, Paul Dirac put forth his equation in it's original form, after arguably abusing operator notation in the Schr{\"o}dinger framework. In the relativistic expression for energy $E=c\sqrt{p^2 + m^2 c^2}$, Dirac replaced $p$ with its operator equivalent to arrive at the following famed equation:

    \begin{equation}
        (\beta m c^2 + c \sum_{n=1}^3 \alpha_n p_n) \Psi(x,t) = i \hbar \frac{\partial \Psi(x,t)}{\partial t}
        \label{The Dirac Equation}
    \end{equation}
}

\subsection{The meaning of the terms used}
{
    \begin{itemize}
        \item $\Psi(x,t)$ is the four-component wavefunction in Hilbert space, and is interpreted as the superposition of a spin-up electron, a spin-down electron, a spin-up positron and a spin-down positron
        \item $m$ is the rest mass of the electron
        \item $p_i$ are the three components of the momentum operator in the Shr{\"o}dinger equation
        \item $c$ is the speed of light
        \item $\hbar$ is the reduced Planck constant, which is usually taken to be unity along with $c$ in natural units
        \item $\alpha_i$ and $\beta$, the terms Dirac introduced, are $4 \times 4$ anticommuting and involuntary Hermitian matrices from Clifford algebra 
    \end{itemize}
}

\subsection{Why the Dirac Equation?}
{
    Even if we ignore the fact that it was the first theory to study relativistic quantum mechanics and paved the way to Quantum Field Theory (QFT) as we know it, the Dirac Equation has a lot to show for. On the surface, it describes all spin $\frac{1}{2}$ massive particles obeying parity conservation, and also provides a proper mathematical framework for studying spin as the inherent angular momentum of the particles. And, of course, the coup de gr{\^a}ce - antimatter! The equation also has corresponding negative energy solutions which puzzled Dirac at first, but his so-called "Hole Theory" led to the positron, and, finally, antimatter as we know it today.
}
