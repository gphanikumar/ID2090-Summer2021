\documentclass[a4paper,12pt]{article}
\usepackage[left=2cm, right=2cm, top=1cm, bottom=2cm]{geometry}
\usepackage{mathtools}
\usepackage{amsmath,esint}
\usepackage{physics}
\usepackage{booktabs}
\usepackage{hyperref}

\begin{document}

\title{My favourite equation from Physics :)}
\author{Janmenjaya Panda}
\maketitle


My favourite equation from physics is none other than the set of 


\textbf{MAXWELL'S EQUATIONS OF ELECTOMAGNETISM}
!!

\begin{center}
	\begin{tabular}{||p{2.5cm}||p{10cm}|p{4cm}||}
	\hline
	\hline
	\multicolumn{3}{|c|}
		{\bf{MAXWELL'S EQUATIONS}} \\
	\hline
	\hline
		Name & Integral Form & Differential Form \\
	\hline
		Gauss Law & 
		$$ \oiint \limits_{\partial\Omega} \overrightarrow{\mathcal E}\cdot d\overrightarrow{\sigma}= \frac{1}{\epsilon_{0}} {\iiint \limits_{\Omega} \rho d\tau} $$ & 
		$$ \div \overrightarrow{\mathcal E}= \frac {\rho}{\epsilon_{0}} $$ \\
	\hline
		No name &
		$$ \oiint \limits_{\partial\Omega} \overrightarrow{\mathcal B}\cdot d\overrightarrow{\sigma}= 0 $$ &
		$$ \div \overrightarrow{\mathcal B}= 0 $$ \\
	\hline
		Maxwell-Faraday law &
		$$ \oint \limits_{\partial\Sigma} \overrightarrow{\mathcal E}\cdot d\overrightarrow{l}= -\frac{\bf{d}}{\bf{dt}} {\iint \limits_{\Sigma} \overrightarrow{\mathcal B} \cdot d\overrightarrow{\mathcal S}} $$ &
		$$ \curl \overrightarrow{\mathcal E}= -\frac{\partial \overrightarrow{\mathcal B}}{\partial{t}} $$ \\
	\hline
		Maxwell-Ampere law &
		$$ \oint \limits_{\partial\Sigma} \overrightarrow{\mathcal B}\cdot d\overrightarrow{l}= \mu_{0} ( \iint \limits_{\Sigma} \bf{\overrightarrow{J}} \cdot d\overrightarrow{\mathcal S} + \epsilon_{0} \frac{d}{dt} {\iint \limits_{\Sigma} \overrightarrow{\mathcal E} \cdot d\overrightarrow{\mathcal S}} ) $$ & 
		$$ \curl \overrightarrow{\mathcal B}= \mu_{0} ( \bf{J} + \epsilon_{0} \frac{\partial \overrightarrow{\mathcal E}}{\partial{t}} ) $$ \\
	\hline
	\hline

	\end{tabular}
\end{center}

\bf{Quantities \& notations :}

\begin{itemize}
	\item $ \overrightarrow{\mathcal E} $ \& $ \overrightarrow{\mathcal B} $ represent electric field vector \& magnetic field pseudovector respectively.
	

	\item $ \bf{\rho} $ \& \bf{ $ \overrightarrow{J} $ } represent volume charge density \& volume current density respectively. 
	

	\item $ \epsilon_{0} $ \& $ \mu_{0} $ represent permittivity \& permeability of free space respectively.
	

	\item $ \div $ corresponds to divergence \& $ \curl $ corresponds to curl.
	

	\item $ \Omega $ is any specified fixed volume with closed boundary surface represented by $ \partial\Omega $.

		Similarily $ \Sigma $ is any specified fixed surface with closed boundary curve represented by $ \partial\Sigma $.
	

	\item $ \oiint \limits_{\partial\Omega} $ corresponds to closed surface integral over boundary surface $ \partial\Omega $ \& $ \iiint \limits_{\Omega} $ is volume integral over volume $ \Omega $.
		
		Similarly $ \oint \limits_{\partial\Sigma} $ corresponds to closed line integral over boundary curve $ \partial\Sigma $ \& $ \iint \limits_{\Sigma} $ is surface integral over surface $ \Sigma $.
	

	\item $ \iiint \limits_{\Omega} \rho d\tau $ is the total electric charge enclosed in volume $ \Omega $.

		Similarly $ \iint \limits_{\Sigma} \bf{\overrightarrow{J}} \cdot d\overrightarrow{\mathcal S} $ is the total electric current passing through the surface $ \Sigma $.
\end{itemize}


\bf{Brief Description :}

Maxwell's equation are a set of differential /integral equations ,which are at the heart of classical electromagnetism ,which essentially describes how electric and magnetic fields are generated by charges ,currents \& changes of fields wrt time.Gauss law describes the interdependence of electrostatic field and electric charges.The net flux of the electric field through a closed surface is essentially propertional to net electric charge inside it with a propertionality constant of $ \frac{1}{\epsilon_{0}} $.The second law is the magnetic analogue of the 1st law and essentially indicates that nothing called magnetic monopole exists in the real physical world.Maxwell-Faraday law indicates how a time vaying magnetic field induces a nonconservative electric field \& Maxwell-Ampere law explains the genesis of magnetic fields,which can be achieved either by electric currents or by time varying electric fields or by both.An important conclusion of these equations are that they demonostrate how fluctuating electromagnetic fields propagate at a constant speed of {\bf{c}} in vaccum ,known as electromagnetic waves.So in the conclusion ,it can be stated that ,at the time of gensis of this Universe, the Supreme Creator,just recapitulated these four equations \& thus the emergence of light as well as all the electromagnetic waves across the spectral series occured !!


\bf{Documentary reference :}


\href{https://en.wikipedia.org/wiki/Maxwell\%27s\_equations}{Maxwell's Equations\_Wikipedia}

\end{document}
