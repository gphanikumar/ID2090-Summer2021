\subsection{Schrödinger Equation}
Formulated by Erwin Schrödinger in 1926. Also called as a analogue of Newton's 2$^{nd}$ law. Schrödinger Equation is elegant and considered to be the "core" of Quantum Mechanics.

Given a set of known initial conditions, Newton's second law makes a mathematical prediction as to what path a given physical system will take over time. The Schrödinger equation gives the evolution over time of a wave function, the quantum-mechanical characterization of an isolated physical system.
$$ \hat{H} \Psi = i \hbar \frac{\partial \Psi}{\partial t} $$
$\Psi$ is Wave function


$\hat{H}$ is Hamiltonian operator.


$\hbar$ is Planck's constant. 


The equation is used extensively in atomic, nuclear, and solid-state physics. The Schrödinger equation describes the form of the probability waves that govern the motion of small particles, and it specifies how these waves are altered by external influences.

"The already ... mentioned psi-function.... is now the means for predicting probability of measurement results. In it is embodied the momentarily attained sum of theoretically based future expectation, somewhat as laid down in a catalog". —Erwin Schrödinger
